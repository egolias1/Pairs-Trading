\documentclass[a4paper,11pt]{article}
\pdfoutput=1 % if your are submitting a pdflatex (i.e. if you have
             % images in pdf, png or jpg format)

\usepackage{jheppub} % for details on the use of the package, please
                     % see the JHEP-author-manual

\usepackage[T1]{fontenc} % if needed



\title{\boldmath Notes on Pairs Trading}


%% %simple case: 2 authors, same institution
%% \author{A. Uthor}
%% \author{and A. Nother Author}
%% \affiliation{Institution,\\Address, Country}

% more complex case: 4 authors, 3 institutions, 2 footnotes
\author{Elliot Golias,\note{Corresponding author.}}
%\author[c]{S. Econd,}
%\author[a,2]{T. Hird\note{Also at Some University.}}
%\author[a,2]{and Fourth}

% The "\note" macro will give a warning: "Ignoring empty anchor..."
% you can safely ignore it.

\affiliation[a]{Case Western Reserve University,\\some-street, Country}
%\affiliation[b]{Another University,\\different-address, Country}
%\affiliation[c]{A School for Advanced Studies,\\some-location, Country}

% e-mail addresses: one for each author, in the same order as the authors
\emailAdd{elliotgolias@case.edu}
%\emailAdd{second@asas.edu}
%\emailAdd{third@one.univ}
%\emailAdd{fourth@one.univ}




%\abstract{Abstract...}



\begin{document} 
\maketitle
\flushbottom

\section{Introduction}
\label{sec:intro}

A basic pairs trading strategy consists of exploiting an out-of-equilibrium market; if two assets typically trade at some spread, then the narrowing/widening of the spread between the assets may be exploited for profit
For example, if the spread widens, then one should buy the low asset and short the high asset. On the other hand, if the spread narrows, than one should short the higher asset and buy the lower asset. 

Consider a state process $\{x_k\}$, where $x_k$ denotes the value of a real variable at the time $t_k = k \tau$, where $\tau$ is the separation between times and $k=0, 1, 2, \dots$. We assume that $\{ x_k \}$ is mean-reverting, meaning we have the following relation between the spread of susequent pairs of $x_k$:
%
\begin{align}
	x_{k+1} - x_k = (a-b x_k) \tau + \sigma \sqrt{\tau} \varepsilon_{k+1},
\end{align}
%
where $\sigma \geq 0, b > 0, a \in \mathbf{R}$, and $\{ \varepsilon_k \}$ is iid Gaussian $\mathcal{N}(0, 1)$ and independent of ${x_k}$. With this definition, the process reverts to $\mu = a/b$ with \textit{strength} $b$. This implies that
%
\begin{align}
	x_k \sim \mathcal{N}(\mu_k, \sigma_k^2),
\end{align}
%
where
%
\begin{align}
	\mu_k = \frac{a}{b} +\left[\mu_0 - \frac{a}{b} \right](1 - b \tau)^k,
\end{align}
%
and
%
\begin{align}
	\sigma_k^2 = \frac{\sigma^2 \tau}{1-(1-b \tau)^2} \left [ 1-(1-b\tau)^{2k} \right] + \sigma_0^2 (1-b \tau)^{2k}
\end{align}
%
We can also express the mean reversion condition in the form
%
\begin{align}
	x_{k+1} = A + B x_k + C \varepsilon_{k+1},
\end{align}
%
where $A = a \tau \geq 0, 0 < B = 1- b \tau$ and $C = \sigma \sqrt{\tau}$.
%
Furthermore, we may also consider the stochastic process $X(k\tau) = x_k$ where $\{X(t), t \geq \}$ satisfies the stochastic differential equation
%
\begin{align}
	\d X(t) = (a - b \ X(t)) \d t + \sigma \d W(t),
\end{align}
%
where $W(t)$ is a standard Brownian motion.

%%%%%%%%%%%%%%%%%%%%%%%%%
\section{Coitegration}
%%%%%%%%%%%%%%%%%%%%%%%%
%
Two time series are cointegrated if some linear combination of both datasets has a constant mean and standard deviation. That is, the time series resulting from the linear combination of the individual










\end{document}
